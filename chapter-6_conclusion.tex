\chapter{Conclusion and Outlook}
\label{chap:conclusion}


%%%%%%%%%%%%%%%%%%%%%%%%%%%%%%%%%%%%%
%%%%%%%%%%%%%%%%%%%%%%%%%%%%%%%%%%%%%
%%%%%%%%%%%%   SECTION   %%%%%%%%%%%%
%%%%%%%%%%%%%%%%%%%%%%%%%%%%%%%%%%%%%
%%%%%%%%%%%%%%%%%%%%%%%%%%%%%%%%%%%%%
\section{Summary}

We started by asking three questions: (1) \textit{What} are the necessary prerequisites in order to achieve efficient rcontrol? (2) \textit{How} do we design and implement this control? (3) \textit{What} are the conditions that allow control to increase total performance?.

Chapter 3 and 4 answer the first question.
%

Chapter 5 aims at answering the remaining two questions on the design and implement of a controller.
%
In the first part, we explained our modifications to the Linux subsystem.
%
To the best of our knowledge, we are the first to implement such a controller. 
%
Building upon this, we described our heuristic approach to minimize control decisions.
%


Our implementation takes into account different capabilities, e.g., data controllability.

Finally, based on our interface, we designed, implemented, and validated our controller.
%


%%%%%%%%%%%%%%%%%%%%%%%%%%%%%%%%%%%%%
%%%%%%%%%%%%%%%%%%%%%%%%%%%%%%%%%%%%%
%%%%%%%%%%%%   SECTION   %%%%%%%%%%%%
%%%%%%%%%%%%%%%%%%%%%%%%%%%%%%%%%%%%%
%%%%%%%%%%%%%%%%%%%%%%%%%%%%%%%%%%%%%
\section{Future Directions}


While we have shown significant gains, our experiment setup is limited and does not cover mixed traffic, different packet sizes, or multiple packet flows and distributions.
%
Therefore, additional measurements are necessary to explore the effects of realistic user traffic.
%