%\addcontentsline{toc}{chapter}{Resume}
\chapter*{Abstract}
\label{chap:abstract}

%
\textit English Abstract 



% correct

In IEEE 802.11 Wireless Local Area Networks (WLANs) a rate adaptation algorithm is employed as a frame transmission rate controller. The Rate controller algorithm Minstrel-HT was developed as the default rate adaptation in the Linux kernel for WiFi standards IEEE 802.11n and 802.11ac, which expanded upon the IEEE 802.11g standard Minstrel algorithm.
Minstrel-HT performs by dynamically selecting data rates based on the current condition of the channel to increase throughput, achieve higher network performance and minimise packet loss~\cite{arif2017evaluation}.

This thesis focuses on analyzing various aspects of network performance and the rate selections made by the Minstrel-HT algorithm. Through this analysis, we gain insights into better optimization strategies. As data transmission rates can vary in unsupervised scenarios compared to supervised experiments, understanding the decision-making process of the Minstrel-HT algorithm is essential. By examining different aspects of the channel environment, we can explore new directions for optimizing the channel.

The thesis introduces a Python-based quantitative analysis tool that enables the examination of monitored data from the Linux kernel's mac80211, providing a comprehensive view of the Minstrel-HT algorithm's decision-making in various scenarios. By observing the results of data transmissions in detail and presenting different analytic plots, we can identify important details that influence transmission rates. This tool also facilitates the exploration of alternative optimization algorithms, allowing for a comparison of their effectiveness in improving transmission rates selection methods.
\newpage

\textit Deutsche Zusammenfassung 


In IEEE 802.11 Wireless Local Area Networks (WLANs) wird ein Algorithmus zur Optimierung der Übertragungsraten eingesetzt. Der Algorithmus Minstrel-HT wurde als Standard Algorithmus im Linux-Kernel für die WiFi-Standards IEEE 802.11n und 802.11ac entwickelt,
als Erweiterung des  IEEE 802.11g-Standard Minstrel-Algorithmus. Minstrel-HT wählt die Datenraten dynamisch, basierend auf dem aktuellen Zustand des Kanals, um den Durchsatz zu erhöhen, die Netzwerkleistung zu erhöhen und Paketverluste zu minimieren [1].
Diese Arbeit zegt die Analyse verschiedener Aspekte der Netzwerkleistung und der Ratenauswahl,
die der Minstrel-HT-Algorithmus vornimmt. Durch diese Analyse erhalten wir Einblicke in bessere Optimierungs
Strategien. 
%Da die Datenübertragungsraten in unüberwachten Szenarien im Vergleich zu überwachten Versuchen variieren können, ist es wichtig, den %Entscheidungsprozess des Algorithmus zu verstehen.
Da die Datenübertragungsraten in unbeaufsichtigten Szenarien im Vergleich zu überwachten Experimenten variieren können, ist das Verständnis des Entscheidungsprozesses des Minstrel-HT-Algorithmus von entscheidender Bedeutung. Durch
Untersuchung verschiedener Aspekte der Kanalumgebung können wir neue Wege zur Optimierung des
des Kanals erschließen.
%In dieser Arbeit wird ein auf Python basierendes quantitatives Analysewerkzeug vorgestellt, das die Untersuchung von Daten aus dem %Linux-Kernel ermöglicht.
Die Arbeit stellt ein Python-basiertes quantitatives Tool vor, das die Analyse aufgezeichneter Daten aus dem mac80211 des Linux-Kernels ermöglicht und einen umfassenden Einblick in die Entscheidungsfindung des Minstrel-HT
Algorithmus in verschiedenen Szenarien erlaubt. Durch detaillierte Beobachtung von Datenübertragungen
und die Darstellung verschiedener analytischer Diagramme können wir wichtige Details identifizieren, 
die die Übertragungsraten beeinflussen. Dieses Tool erleichtert auch die Untersuchung alternativer Optimierungsalgorithmen und ermöglicht einen
Vergleich ihrer Effektivität bei der Verbesserung der Auswahlmethoden für die Übertragungsraten.






