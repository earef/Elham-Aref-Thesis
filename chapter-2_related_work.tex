%%%%%%%%%%%%%%%%%%%%%%%%%%%%%%%%%%%%%%%%%%%%%%%%%%%%%%%%%%%%%%%%%%%%%%%%%%
%%%%%%%%%%%%   CAPTER 2   %%%%%%%%%%%%%%%%%%%%%%%%%%%%%%%%%%%%%%%%%%%%%%%%
%%%%%%%%%%%%%%%%%%%%%%%%%%%%%%%%%%%%%%%%%%%%%%%%%%%%%%%%%%%%%%%%%%%%%%%%%%
\chapter{Related Work}
\label{chap:related_work}

%intro
%
To this end, we start with a short overview and then present the state-of-the-art of current approaches. Next, we discuss the current work.
%
Finally, we conclude with a summary and open problems.


%%%%%%%%%%%%%%%%%%%%%%%%%%%%%%%%%%%%%
%%%%%%%%%%%%%%%%%%%%%%%%%%%%%%%%%%%%%
%%%%%%%%%%%%   SECTION   %%%%%%%%%%%%
%%%%%%%%%%%%%%%%%%%%%%%%%%%%%%%%%%%%%
%%%%%%%%%%%%%%%%%%%%%%%%%%%%%%%%%%%%%
\section{A Short Overview}
\label{s:basics}

%\begin{figure}
%	\center
%	\includegraphics[width=\columnwidth]{figures/overview}
%	\caption{Overview.}
%	\label{fig:overview}
%\end{figure}


%%%%%%%%%%%%%%%%%%%%%%%%%%%%%%%%%%%%%
%%%%%%%%%%%%%%%%%%%%%%%%%%%%%%%%%%%%%
%%%%%%%%%%%%   SECTION   %%%%%%%%%%%%
%%%%%%%%%%%%%%%%%%%%%%%%%%%%%%%%%%%%%
%%%%%%%%%%%%%%%%%%%%%%%%%%%%%%%%%%%%%
\section{Control theory}
\label{s:single_control}

%
In this section, we discuss these approaches in their separate sections.
%

%%%%%%%%%%%%%%%%%%%%%%%%%
%%%%%   SUB-SECTION   %%%
%%%%%%%%%%%%%%%%%%%%%%%%%
%%%%%%%%%%%%%%%%%%%%%%%%%
\subsection{Old Control}
\label{ss:oc}


%summary table of old protocols
\begin{landscape}
\ctable[
	cap	= Summary of old Control (OC) Algorithms,
	caption = Summary of old Control (OC) Algorithms,
	label	= tab:oc_summary,
	pos  	= h
]{lccccc}{
%
\tnote[1]{Abbreviations: Obj: Objective, E: Energy, T: Topology, C: Capacity, Dist.: Distributed, Cent.:Centralized, synch: Synchronized, P: Prototype}
\tnote[2]{Old conditions are similar in space and time.}
\tnote[3]{Ideal comm: Overhearing any transmissions.}
\tnote[4]{New control considered.}
\tnote[5]{Angles between 5\textdegree to 120\textdegree are considered.}
%
} {
	\FL
	%
	\textbf{Protocol}			& \textbf{Obj.}\tmark[1]	& \textbf{Granularity}	& \textbf{Type of Control} & \textbf{PHY layer assumptions} & \textbf{Validation}\ML
	%
        PAR~\cite{Chen02-WNJ}			& E	& Per-packet	& Dist.	& Sym\tmark[2], ideal comm.\tmark[3],omni\tmark[4] & \no	\NN
        \hline	
	TER~\cite{154a}		& T	& Per-network	& Cent.	& Sym, ideal comm.,omni & \yes (Routing)	\NN
	MLP~\cite{aazhang88}		& T	& Per-cluster	& Cent.	& Sym, ideal comm.,omni & \yes (Routing)	\NN
        \hline	
	PCM~\cite{FuInfocom03}			& C	& Per-packet	& Dist.	& Sym, ideal comm., omni & \no	\NN
	POW~\cite{aiello03}	& C	& Per-packet	& Dist.	& Sym, ideal comm., omni & \no	\NN
	MID~\cite{callaway02}	& C	& Per packet	& Dist.	& Directional\tmark[5]	& \yes (MAC,WARP)	\NN
	DIR~\cite{bianchi00}			& C	& Per slot	& Cent.	& Directional	& \yes (MAC,Madwifi)	\NN
	SPE~\cite{crow97}			& C	& Per slot	& Cent.	& Directional	& \yes 	(Multi-radio)	\LL
}
\end{landscape}


In summary, it can be concluded that control does not bring energy savings.


%%%%%%%%%%%%%%%%%%%%%%%%%
%%%%%   SUB-SECTION   %%%
%%%%%%%%%%%%%%%%%%%%%%%%%
%%%%%%%%%%%%%%%%%%%%%%%%%
\subsection{New Control}
\label{ss:trc}

In this section, we summarize the algorithms in literature.
%
Table~\ref{tab:oc_summary} summarizes the approaches considered in this section.
%


%%%%%%%%%%%%%%%%%%%%%%%%%%%%%%%%%%%%%
%%%%%%%%%%%%%%%%%%%%%%%%%%%%%%%%%%%%%
%%%%%%%%%%%%   SECTION   %%%%%%%%%%%%
%%%%%%%%%%%%%%%%%%%%%%%%%%%%%%%%%%%%%
%%%%%%%%%%%%%%%%%%%%%%%%%%%%%%%%%%%%%
\section{Summary}

In this section, we gave an overview of the research on both theoretical and practical control.
%


